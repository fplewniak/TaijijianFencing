% !TeX spellcheck = fr_FR
\chapter{La pratique du \Taijijian{}}

\Taijijian{} est la transcription en pinyin du mot chinois désignant l'art de l'épée basé sur les principes du  \Taiji{} \textendash{} aussi connu sous le nom d'épée du \Taiji{} (ou épée du Taichi) \textendash{}  qu'on étudie en parallèle du \Taijiquan{}, la boxe du \Taiji{}.

Dans la plupart des styles, la pratique du \Taijijian{} consiste essentiellement, sinon exclusivement, à apprendre et dérouler une forme à l'épée. En tant que tel, il complète la pratique à mains nues et apporte aux pratiquants un outil inestimable pour améliorer leur pratique.
L'épée est en effet une partenaire dévouée, toujours prête à nous guider sur la voie d'une meilleure compréhension et incarnation des principes du \Taiji{}.

Cependant, je suis convaincu que, dans cette optique, la pratique de l'épée du \Taiji{} ne peut que bénéficier de l'étude approfondie des techniques de bases, des notions d'escrime, d'exercices avec partenaire, des applications martiales et même d'assauts libres.

Au XXI\textsuperscript{e} siècle, l'escrime du \Taiji{} a perdu sa dimension appliquée: la dextérité au combat et l'efficacité ne sont plus des objectifs en soi mais leur but n'est autre que l'application dans l'action des principes du \Taiji{}.
Des techniques de base jusqu'à l'assaut libre, les pratiquants s'attacheront à ne faire qu'un avec leur épée, à améliorer leur agilité et leur sensibilité, à relâcher leur corps et leur esprit, à développer leur \Yi{}, etc.

Enfin, et ce n'est pas le moindre, cette entreprise de toute une vie devrait aussi être simplement plaisante.

\section{\index{exercices de base}Exercices de base et \index{echauffement@échauffement}échauffement}
Le but de l'échauffement et des exercices de base est de préparer le corps et l'esprit à la réalisation de techniques efficaces en toute sécurité et dans le respect des principes du \Taiji{}.

L'échauffement doit tranquillement mobiliser les articulations et les muscles pour les amener à leur niveau optimal de fonctionnement pour assurer des mouvements fluides et réduire les risques de traumatismes.
L'accent doit être mis sur la partie supérieure du corps, en particulier les épaules, les bras, les poignets et les doigts qui sont très sollicités dans la tenue de l'épée.
La partie inférieure du corps ne doit toutefois pas être délaissée en préparation des déplacements qui sont d'une importance primordiale en escrime.

J'ai habituellement tendance à commencer l'échauffement en mobilisant le bassin et l'axe vertébral qui sont au c\oe{}ur de tous les mouvements de \Taiji{}, puis je continue avec les épaules, les coudes, les poignets et les doigts.
Pour la partie inférieure du corps, je commence avec les chevilles avant de passer aux genoux puis aux hanches, exerçant l'équilibre par la même occasion.
La séance s'achève alors avec des déplacements codifiés et libres.

Avant de passer aux techniques de base, il est judicieux de finir l'échauffement en utilisant l'épée comme accessoire. Cela contribuera non seulement à étirer plus en profondeur les poignets et épaules, cela aidera également les pratiquants à installer une relation avec leur épée.
La répétition de mouvements simples et non techniques aide en effet à relier l'épée et le corps: l'énergie de l'épée, absorbée et concentrée dans le corps, est renvoyée à l'épée pour le mouvement suivant\footnote{Voyez le chapitre \ref*{ch:prise} pour plus de détails sur la manipulation de l'épée.}.

La répétition est également d'une importance fondamentale pour la pratique des techniques de base, les emblématiques coupes et estocs élémentaires. Les répéter en séries est essentiel non seulement pour la précision technique et le contrôle du corps, c'est aussi un préalable nécessaire à la réalisation correcte des techniques  dans des situations moins contrôlées.

Il est intéressant également de pratiquer les techniques de base avec des \index{cibles}cibles, non seulement pour exercer la précision, mais aussi pour développer l'état d'esprit et la relaxation du corps nécessaires à leur réalisation sans effort.
Pour les \index{estocs}estocs, \Ci{} ou \Zha{}, des pinces à linge accrochées à un fil à hauteur de gorge font des cibles très convenables.
Les cibles pour les \index{coupes}coupes sont plus difficiles à réaliser: les coupes étant censées passer à travers la cible, il n'est pas question d'utiliser des cibles dures et résistantes. Des tiges de plantes non ligneuses ou un pain de terre glaise peuvent constituer des cibles de coupe intéressantes et bon marché, ne nécessitant de surcroit pas une épée dangereusement affûtée.
La glaise convient très bien à la coupe avec une épée de pratique habituelle et non tranchante, mais je déconseille l'utilisation de votre épée favorite à moins d'être prêt à la nettoyer dans les moindre recoins après la séance.
Il est bon également de s'assurer que la terre glaise n'est pas souillée par des gravillons ou du sable afin d'éviter d'abimer la lame. 

Ces exercices permettent d'acquérir les bases qui, combinées entre elles, constituent la source de toutes les techniques pratiquées dans le contexte plus diversifié de la forme.


\section{\index{forme!pratique@\textit{pratique}}Forme}
La forme est un ensemble de mouvements arrangés en une suite continue de techniques que certains pratiquants présentent comme un simulacre de combat avec un adversaire imaginaire.

Je pense pour ma part qu'il y a plus : bien que les mouvements de la forme soient en effet des techniques martiales, l'ensemble ne représente pas un unique combat du début à la fin. Les techniques sont plutôt arrangées en courtes séquences, que la tradition européenne désigne sous le nom de \textit{pièces}, décrivant une variété de situations dans lesquelles ces techniques peuvent être appliquées. Des variations sont présentes tout au long de la forme et chaque mouvement pouvant être appliqué de multiples manières, les séries peuvent se chevaucher ou décrire des situations différentes.

La forme est bien plus qu'un simple catalogue de techniques, c'est la source de laquelle découlent toutes les applications en fonction des situations possibles. En tant que réalisation du principe \index{YinYang@\Yin{}/\Yang{}}\Yin{}/\Yang{} né du \Taiji{}, la forme de \Taijijian{} contient toutes les potentialités de réalisation d'innombrables applications. Pratiquer la forme revient donc à générer le potentiel d'une créativité infinie, réservant l'expression pratique des techniques pour leur application en situation réelle.

\`{A} mon avis, la forme est donc essentiellement un outil permettant d'acquérir une meilleure compréhension et incarnation des principes du \Taiji{}.
La mémorisation n'est qu'un début : ce à quoi on s'exerce vraiment en pratiquant la forme est la \index{transformation}transformation \index{YinYang@\Yin{}/\Yang{}}\Yin{}/\Yang{} et la conduite du \index{Yi@\Yi{}}\Yi{} en un flot continu et toujours changeant. 
Petit à petit, avec la pratique, on atteint l'unité avec l'épée, la forme devient de plus en plus interne, gagne en fluidité, semble aisée et naturelle.
Jusqu'à ce que, me plais\textendash{}je à penser, le corps et l'esprit soient finalement délivrés de toutes leurs tensions et que rien ne subsiste qu'un pur \index{Yi@\Yi{}}\Yi{} générant la forme sans effort.

La forme n'est cependant pas uniquement un exercice mental, elle entraîne aussi physiquement le corps. 
L'amplitude exagérée de certains mouvements permet en effet de développer la puissance et la souplesse.
D'autres ont clairement comme objectif d'être spectaculaires et démonstratifs.
Toute forme de \Taijijian{} est ainsi faite d'un mélange d'entraînement martial interne, d'exercice gymnique, et de spectacle.
Percevoir comment ces caractères sont en réalité exprimés dans les différents mouvement permet au pratiquant de privilégier à volonté l'un ou l'autre aspect. 

Ceux qui sont intéressés par l'amélioration de leur habileté à l'escrime trouvent dans la forme un outil inestimable pour se forger un répertoire et une connaissance solides de techniques martiales, ainsi que l'occasion de pratiquer la mécanique corporelle adéquate pour leur application effective.

La forme n'est toutefois pas exactement représentative des assauts libres pendant lesquels les techniques les plus simples sont les plus fréquentes et les plus efficaces
La forme propose des mouvements étonnamment complexes qui peuvent contribuer à son caractère démonstratif, mais préparent probablement aussi le pratiquant à mettre en \oe{}uvre des techniques extrêmes, adaptées à des circonstances exceptionnelles.
La plupart des bretteurs de l'époque n'ont certainement jamais eu à mettre ces techniques en pratique, mais ceux qui ont vraiment eu à le faire ont peut-être dû leur vie à cette préparation au lieu de se trouver stupéfaits devant une situation inattendue

\`{A} notre époque moderne, l'efficacité martiale de l'escrime du \Taiji{} n'est plus une question de survie, mais, dans le contexte d'une pratique amicale et contrôlée, elle vise plutôt à libérer l'esprit et le corps de leurs tensions.
L'interprétation martiale de la forme propose donc toute une série de situations où les pratiquants peuvent mettre leur corps et leur esprit à l'épreuve dans le cadre d'exercices avec partenaire et d'applications martiales.

\section{\index{exercices avec partenaire!pratique@\textit{pratique}}Exercices à deux et \index{applications martiales!pratique@\textit{pratique}}applications martiales}
Les exercices avec partenaire sont des mises en situation simplifiées et codifiées dont le but est d'introduire et d'exercer les notions et principes fondamentaux de l'escrime du \Taiji{}, tels que la distance et le temps, les lignes, le sentiment du fer, les déplacements selon ceux du partenaire, l'agilité, etc.
Entièrement dédiés à cet objectif pédagogique, il s'agit essentiellement d'exercices continus ou de jeux, le plus souvent sans aucune préoccupation de réalisme.

Les applications martiales développent ensuite les mêmes principes et notions à l'aide de pièces codifiées ou semi-codifiées, illustrant par la même occasion l'utilisation pratique potentielle des techniques de la forme.
Elles ouvrent ainsi la voie vers une meilleure compréhension de celle-ci, mettant en lumière l'essence martiale des mouvements et introduisant une distinction entre gestes pratiques et démonstratifs.

Bien que les applications martiales soient plus réalistes que les exercices avec partenaire, il faut garder à l'esprit qu'elles n'en sont pas moins des simulations, très loin de pouvoir reproduire toutes les conditions d'un véritable duel à mort.
Selon leur expérience et les protections qu'ils portent, les pratiquants réaliseront les applications à des vitesses différentes ou se montreront plus ou moins coopératifs envers leur partenaire, atteignant ainsi des niveaux différents de réalisme.
Certaines applications peuvent fonctionner très bien à faible vitesse, avec un partenaire attentionné,  mais peuvent tout à fait ne pas fonctionner à plus grande vitesse, avec un partenaire ne retenant pas son attaque.
L'utilisation de \index{protections}protections complètes permettant à l'attaquant d'aller à la touche rend donc les applications martiales plus difficiles et ce qui semblait fonctionner avec moins de protections et plus de précautions peut très bien ne plus aussi bien marcher.   
\`{A} l'opposé, réaliser une application martiale lentement risque de laisser à un partenaire non coopératif l'occasion de contre-attaquer alors qu'une telle riposte demanderait d'être plus rapide que l'éclair contre une application à pleine vitesse. 

L'évaluation de l'efficacité réelle des applications martiales demande donc de prendre en compte tous ces facteurs.
Quoiqu'il en soit tant qu'elles nous permettent de développer et de pratiquer les principes du \Taiji{} et les notions d'escrime, le réalisme approximatif des applications martiales conviendra parfaitement à notre objectif.
L'efficacité d'une application devrait donc être seulement la conséquence du respect des principes et certainement pas une question de vitesse ni de force.
Ainsi, la pratique des applications martiales nous amène à développer l'état d'esprit et la disposition corporelle appropriés pour la mise en application effective des techniques et principes en assaut libre.

\section{\index{assaut libre!pratique@\textit{pratique}}Assaut libre}
L'assaut libre désigne la simulation non codifiée d'un duel à l'épée.
En fonction des protections que portent les pratiquants et de leur expérience, on peut ajouter des règles pour garantir leur sécurité.
Quoiqu'il en soit, la violence est totalement exclue et l'assaut libre doit toujours rester un jeu amical, sans excès d'esprit de compétition.

Je comprends que certains pratiquants s'inquiètent de ce que l'assaut libre pourrait ne pas être une pratique interne. 
En réalité, il ne faut pas confondre expression des techniques martiales et pratique externe.
\`{A} mon avis, une caractéristique de \index{pratique interne}l'interne est que chaque mouvement nait du \index{Yi@\Yi{}}\Yi{} qui donne forme à la technique et, à partir du centre est finalement exprimé vers la périphérie. Les techniques efficaces découlent naturellement d'une intention appropriée et d'un corps relâché, des principes bien maîtrisés qui, devenus naturels, peuvent ainsi être appliqués spontanément.
Les étudiants des arts internes ne devraient pas trop s'attacher aux détails techniques insignifiants qui ne sont que le doigt du sage.
Il leur faut chercher à atteindre la lune : à développer leur capacité à appliquer les principes dans des situations imprévisibles et difficiles.\footnote{Pour être honnête, je ne pense pas qu'il y ait tant de différences entre interne et externe à un niveau élevé de pratique. La différence principale serait plutôt liée à la manière dont ces arts sont abordés Les arts externes se focalisent d'abord sur les techniques et laissent à l'étudiant le soin de découvrir les principes tandis que les étudiants d'interne étudient les principes et doivent ensuite découvrir comment réaliser les techniques en les respectant.}
La technique suivra. 

L'agilité du \index{Yi@\Yi{}}\Yi{} et du corps résulte d'un esprit ouvert et libre, sans tension, nous permettant de rester relâchés face aux menaces d'un adversaire.
Après tout, ce pourrait bien être la véritable application des arts martiaux à notre époque : ne pas se laisser submerger par le stress dans les situations d'urgence.

Bien entendu, comme mentionné plus haut, l'efficacité des techniques dépend entièrement du contexte : l'assaut libre est \textendash{} et doit rester \textendash{} une simulation qui ne peut en aucun cas reproduire tous les aspects d'un véritable duel à l'épée, et en particulier les aspects psychologiques.

Quoiqu'il en soit, de nos jours, les désaccords ne se règlent plus en duel et le but de l'escrime du \Taiji{} est plus une affaire de développement personnel que de réelles aptitudes au combat. Notre préoccupation principale n'est pas de frapper l'adversaire à tout prix, mais de le faire de la manière appropriée sans être soi-même touché. La façon dont le but est atteint est plus importante que le but lui même : marquer une touche contre un adversaire devrait être le résultat de l'application correcte des principes du \Taiji{} à la situation et non pas un objectif en soi.

Ainsi, pour éviter un excès de compétition, je préfère ne pas compter les touches et ne faire qu'apprécier subjectivement la qualité des actions.  L'enregistrement vidéo des assauts libres peut aussi aider à passer ultérieurement en revue les actions pour en identifier les points positifs et négatifs. 
En l'absence d'enjeu, cette approche moins compétitive permet de se concentrer plus sur les principes et la pratique interne tout en limitant les risques d'accident. 


\section{\index{sécurité}Considérations de sécurité}
La pratique du \Taiji{} est associée à la préservation de la santé et au développement personnel depuis peut-être plus d'un siècle.  On peut donc s'attendre à ce que conserver une bonne santé aille de paire avec la préservation de notre intégrité physique, et donc que la pratique du \Taijijian{}, de la forme à l'assaut libre, puisse être entreprise en relative sécurité.
En fait, même l'entraînement rude des guerriers des temps passés a pu présenter un certain degré de sécurité : à quoi bon décimer les troupes avant même de les envoyer au combat?

Bien sûr, il y a parfois des accidents, mais il n'y a aucune raison pour laquelle ils devraient être la norme. 

Il nous faut en toutes circonstances garder à l'esprit que même une épée non affûtée reste une arme qui peut se révéler mortelle et devrions nous comporter en conséquence. 
Pratiquer en sécurité est en réalité le résultat de la combinaison d'une attitude responsable et de l'équipement adéquat. 
L'état d'esprit que nous adoptons pendant la pratique, quelle qu'elle soit, est en effet des plus importantes. Il est à mon avis essentiel de se sentir responsable non seulement de l'intégrité physique des personnes qui nous entourent lorsque nous tenons ou manipulons une épée, mais aussi de notre propre sécurité. 
La conséquence principale de cet état d'esprit est que, lorsque tous les pratiquants l'adoptent, chacun maintient en permanence un certain degré de vigilance au lieu de s'en remettre entièrement aux autres pour assurer la sécurité de tous. 
De plus, si jamais nous sommes tout de même blessés malgré toutes les précautions prises, cette attitude nous évite de systématiquement en reporter la faute sur les autres. 

Il est également important de ne pas oublier que la pratique en solo n'est pas exempte de danger.  Notre préoccupation pour la sécurité ne doit pas se limiter au lieu de pratique mais commence dès le vestiaire. 
Dès que nous avons l'épée à la main ou que nous la manipulons, il nous faut le faire avec beaucoup de précaution. En portant une épée, on doit en tenir la lame verticalement le long du bras ou la pointe dirigée vers le sol. Il faut impérativement éviter agiter inconsidérément une épée et toujours s'assurer d'être à bonne distance des autres avant de commencer à pratiquer. 

Quant aux \index{exercices avec partenaire!sécurité@\textit{sécurité}}exercices avec partenaire, aux \index{applications martiales!sécurité@\textit{sécurité}}applications ou à \index{assaut libre!sécurité@\textit{sécurité}}l'assaut libre, il faut toujours adapter la vitesse et l'intensité de la pratique au partenaire le moins avancé et au matériel de protection utilisé. 
 
Il est préférable que les deux partenaires utilisent le même type d'équipement : une épée mouchetée\footnote{Voyez le chapitre \ref*{ch:epeechinoise} pour de plus amples détails.} et un \index{protections}masque d'escrime sont un minimum. Des gants et une veste rembourrée permettent un travail plus dynamique et sont vivement recommandés en l'assaut libre.  

Suivre les conseils ci-dessus permet de réduire les risques d'accident à leur minimum. Toutefois, il faut ne pas oublier que toute pratique martiale présente un risque inhérent, qu'un accident peut arriver et que dans ce cas, on ne peut qu'en limiter les conséquences grâce à l'attitude et à l'équipement appropriés. Tout pratiquant d'escrime du \Taijijian{} se doit de se faire à cette idée et d'accepter qu'il s'engage dans cette voie à ses risques et périls.
