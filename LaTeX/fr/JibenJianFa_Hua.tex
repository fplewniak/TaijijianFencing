% !TeX spellcheck = fr_FR
\section{\Hua}
Le verbe \Hua{} \begin{CJK*}{UTF8}{bsmi}劃\end{CJK*} signifie \textit{délimiter}, \textit{tracer}. Ce caractère est aussi une variante du mot désignant un trait dans un caractère chinois.
La partie de gauche de ce caractère étant la clé du pinceau, ces observation suggèrent que cette technique évoque la notion de calligraphie, d'écriture et de dessin.

Le \Yangjia{} \Michuan{} présente la technique \Hua{} emblématique comme une coupe horizontale ou un large mouvement horizontal dans le but de maintenir les adversaires à distance. L'idée est ici de balayer l'espace avec l'épée pour délimiter la zone la plus large possible autour de soi et d'entailler quiconque ose s'approcher.

D'une manière plus générale, les coupes \Hua{} ne sont pas systématiquement horizontales et s'appliquent à différentes distances, depuis des entailles réalisées à longue distance avec la pointe de l'épée, jusqu'à des coupes glissées avec toute la longueur du tranchant à courte distance. Dans tous les cas, les coupes \Hua{} ont pour caractéristique commune le fait que la lame trace progressivement une longue entaille, au contraire de \Pi{} qui fend la cible d'un coup. 

Pour réaliser une coupe \Hua{} à longue distance, l'épée est lancée en avant et, lorsque le bras a presque atteint son extension complète, juste avant que la lame ne frappe la cible, la prise se resserre doucement sur la poignée pour assurer la connexion entre les centres de l'épée et du corps. La rotation continue ainsi à partir de l'épaule tandis que l'épée tire le corps en avant jusqu'à ce que la portée maximale soit atteinte (fig. \ref{fig:hua_cut} a-c). Ensuite, la prise agissant comme un levier, l'inertie de l'épée repousse la poignée contre le talon de la main, engendrant un mouvement vers l'arrière qui génère une coupe glissée et recentre le corps dans une posture de garde (fig. \ref{fig:hua_cut} d-f). 

\begin{figure}[ht]
\centering
	\includegraphics[width=1.00\textwidth]{../../Images/JibenJianfa/Hua/Hua.pdf}
	\caption[Coupe \Hua{}]{Coupe \Hua{} : (a) \'{A} partir d'une position haute de l'épée, (b) la main droite lance le pommeau vers l'avant; (c) montre la fin de la phase active de la technique; (d) à (f) pendant la phase passive, l'inertie de l'épée repousse la main en arrière, réalisant la coupe glissée et recentrant le corps en position.}
	\label{fig:hua_cut}
\end{figure}

\'{A} plus courte distance, la dynamique des coupes \Hua{} utilise moins l'inertie de l'épée mais s'appuie plus sur la structure et le mouvement du corps. Dès que le tranchant est en contact, la coupe est effectuée en pressant le tranchant contre la cible et en tirant l'épée selon une direction parallèle à l'épée, en se déplaçant ou en tournant le corps. Il est parfois possible, en particulier en passant dans le dos de l'adversaire, de réaliser avec le faux tranchant une coupe \Hua{} à courte distance.

En plus d'être une coupe glissée, \Hua{} peut aussi être utilisé pour maintenir des adversaires à distance ou les obliger à réagir de façon à exploiter leur action et prendre le contrôle du rythme. Pour ce faire, on peut lancer une coupe \Hua{} sans toutefois s'engager totalement, ou faire des moulinets tout en avançant. La distance doit dans ce cas est suffisamment courte pour que ces coupes soient clairement perçues comme une menace bien qu'on puisse être légèrement hors de mesure. La distance idéale est à la limite supérieure de la mesure courte, distance à laquelle, bien qu'elle soit incertaine, une touche est tout à fait plausible et l'adversaire ne peut que se sentir obligé de réagir défensivement. Il est important dans ce cas de se préparer à doubler avec une attaque plus engagée ou un contrôle de la lame selon les circonstances. C'est cette seconde intention qui permettra de prendre l'initiative en exploitant l'action de l'adversaire.

En rompant la distance après une attaque infructueuse, il est possible de maintenir l'adversaire à distance avec une série de moulinets que Maître Wang Yennien décrivait également comme étant des coupes \Hua{}. Cette application de la technique correspond parfaitement à la traduction \textit{délimiter} puisqu'elle crée en effet une zone de sécurité empêchant l'adversaire d'entrer et d'attaquer pendant qu'on se place hors de mesure.
