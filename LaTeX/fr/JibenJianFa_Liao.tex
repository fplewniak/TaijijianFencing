% !TeX spellcheck = fr_FR
\section{\Liao}
\Liao{} \begin{CJK*}{UTF8}{bsmi}撩\end{CJK*}, qu'on peut traduire par \textit{lever}, \textit{soulever}, \textit{asperger}, est fréquemment mentionné dans les manuels d'arts martiaux pour de nombreuses armes. La tradition du \Yangjia{} \Michuan{} décrit cette technique comme une coupe montante, mais aussi parfois comme une action défensive utilisée pour parer ou dévier une attaque. Elle est souvent présentée en séries de deux coupes consécutives ou comme une parade immédiatement suivie d'une coupe ascendante.

Il n'est pas précisé s'il s'agit d'un fendant ou d'une entaille mais il semble que les deux options soient également possibles. Les détails techniques varient selon le type de coupe ou que \Liao{} est utilisée pour parer. Toutes les variations ont cependant en commun la direction ascendante du mouvement, comme pour écarter un rideau. Ceci peut expliquer pourquoi cette \Jianfa{} est désignée par un terme décrivant le mouvement général, \textit{soulever}, plutôt que la coupe elle-même au contraire de \Ci{}, \Pi{}, \Duo{}, et \Hua{}. En faveur de cette hypothèse, on peut noter également que le caractère \Liao{} \begin{CJK*}{UTF8}{bsmi}撩\end{CJK*} comporte la clé de la main au lieu de celle du couteau.

La technique \Liao{} de base est réalisée en levant la main en diagonale depuis une ligne basse vers la ligne haute opposée conjointement avec une rotation de la taille. (fig)
Selon le type de coupe, entaille (type \Hua{})  ou fendant (type \Pi{}), ou si \Liao{} est combiné avec \Mo{} ou un battement, le mouvement de la main sera plus ou moins étendu vers l'avant. Alors qu'une déflexion implique un coup dégageant sèchement la lame de l'adversaire, un engagement plus doux permet de contrôler et de guider délicatement son épée vers le haut pour dégager le chemin tout en conservant notre propre pointe vers son visage pour préparer la riposte. Dans ce cas, il s'agit de prendre d'abord le contact avec la lame de l'adversaire, puis de soulever l'épée tout en suivant et contrôlant son attaque. En combinant ainsi les techniques \Liao{} et \Mo{}, le contact des lames est primordial pour préparer une réponse appropriée à l'attaque de l'adversaire avec le minimum de risque. Ce contact n'est pas uniquement une protection mais permet aussi de percevoir et suivre les réactions de l'adversaire pour les tourner à notre avantage.

Les coupes \Liao{} de base se font en oblique, d'une ligne basse vers la ligne haute opposée, la lame étant alignée avec le plan de coupe et frappant la cible dans le plan sagittal. \`{A} la fin de la coupe, le relâchement de la tenue permet de recentrer la pointe de la lame avec l'autre plan de coupe pour préparer une nouvelle coupe \Liao{}. Cette transition se fait en avant du corps de façon à rester protégé derrière la lame plutôt qu'en faisant un large mouvement ouvert.

Il existe une variant de coupe \Liao{} réalisée en esquivant une attaque avec un pas de côté tout en soulevant son épée pour intercepter le bras de l'adversaire. Alors que l'interception se fait dans le plan sagittal après être sorti de la ligne d'attaque, la coupe est en fait une entaille de type \Hua{} plutôt latérale, réalisée avec une passe avant dans la même direction que l'esquive initiale.

Un cas extrême de coupe  \Liao{} latérale est réalisée entièrement dans le plan frontal, de la gauche vers la droite. Elle commence avec une rotation de la taille vers la gauche pour esquiver ou dévier une attaque en préparant la coupe. Ceci peut se faire soit sur place ou avec un pas croisé (demi-volte) pour ajuster la distance et la direction. Dans les deux cas, on finit avec l'adversaire à notre droite et on envoie l'épée vers la droite tout en s'accroupissant pour lancer une coupe\Liao{} de type \Pi{} visant le bras, l'entrejambe, etc. Il est important alors de tourner légèrement le corps vers la droite en s’accroupissant de façon à libérer l'épaule droite et permettre de lever l'épée suffisamment haut pour atteindre la cible et la fendre.
