% !TeX spellcheck = fr_FR
\section{\Ci{}}
On trouve le mot \Ci{} \begin{CJK*}{UTF8}{bsmi}刺\end{CJK*}, signifiant \textit{estoquer}, \textit{percer}, \textit{poignarder}, dans le \WubeiZhi{} comme terme générique désignant l'ensemble des techniques d'estoc à l'épée. D'autres traités anciens le mentionnent aussi pour décrire les estocs avec diverses armes. 

Dans la tradition du \Yangjia{} \Michuan{}, \Ci{} est défini comme un estoc horizontal ou remontant, poussant la pointe puissamment à travers la cible. 

On réalise généralement la technique formelle en commençant sur le pied droit, pied gauche en avant, soit avec une passe avant (\Ci{} long), soit avec un simple transfert de poids sur le pied gauche (\Ci{} court). Dans le contexte formel des exercices et de la forme, la cible du \Ci{} court est l'abdomen, celle du long est la base de la gorge. En assaut par contre, on peut viser d'autres cibles telles que le torse, ou même le visage.

Que la technique réalisée soit longue ou courte, on commence invariablement par créer dans le corps une structure spiralée connectant le pied gauche et l'épée. Dès que la taille bouge, le bras droit pousse sur la poignée et se lève en un mouvement spiralé qui s'achèvera avec une position d'épée horizontale, sur son plat, le pommeau orienté vers la hanche gauche. Dans le même temps, on transfère le poids sur le pied gauche. La prise de l'épée s'adapte progressivement pour maintenir une connexion ininterrompue entre la main et la poignée, sans aucun angle, permettant de pousser sans effort l'épée vers avant. Cet ajustement permet également d'exercer sur la poignée une action oblique atteignant au delà de la garde le point générant un point pivot dans la pointe de la lame pour la stabiliser\footnote{Voyez le chapitre \ref*{ch:epeechinoise} pour de plus amples détails sur les points pivots.}. 


La passe avant du \Ci{} long permet une portée plus longue. Le bras droit doit être étendu avant d'avancer pour augmenter la précision de l'estoc et placer le corps derrière l'épée, aussi loin que possible du danger. Engager la lame de l'adversaire pour la contrôler avec la garde ou le fort de la lame, permet encore une meilleure protection. 

\begin{figure}[ht]
\centering
	\includegraphics[width=1.00\textwidth]{../../Images/JibenJianfa/Ci/Ci_thrust_arc.pdf}
	\caption[Estoc \Ci{} long]{\`{A} la fin de l'estoc \Ci{} long, l'épée est alignée avec la hanche gauche mais sa pointe est dirigée vers le centre, visant la base de la gorge. La puissance de toute la structure du corps se concentre dans l'épée pour pousser la pointe à travers la cible.}
	\label{fig:ci_thrust}
\end{figure}

Idéalement, le talon droit devrait toucher le sol exactement au moment où la pointe de la lame atteint la cible. Le relâchement de la structure achève alors le pas en poussant la lame au travers de la cible. Il est important de ne pas se laisser tomber dans la jambe droite de manière à conserver une capacité de se retirer rapidement en cas de nécessité. Cela ne signifie toutefois pas qu'on ne doit jamais transférer le poids sur la jambe droite, mais que la polarité vide/plein entre les deux jambes doit être maintenue en toutes circonstances pour éviter la double lourdeur. L'arc de force venant du pied gauche, traversant le dos, spiralant le long du bras droit jusqu'à la pointe de l'épée, et soutenu par la spirale du bras gauche, génère ainsi une structure à la fois puissante et mobile.
