% !TeX spellcheck = fr_FR
\chapter{Avant-propos}\label{avant-propos}

\paragraph*{}
Dans la plupart des styles et des écoles, le \Taijijian{} est presque exclusivement étudié sous la forme d'enchaînements, le plus souvent sans se soucier particulièrement de ses racines martiales.
Bien que l'on rencontre aujourd'hui un intérêt croissant pour des exercices avec partenaire, principalement d'épées collantes, les applications martiales sont très rarement présentées et lorsqu'elles le sont, se limitent souvent à des explications et justifications martiales des mouvements de la forme.
Ce qu'on pourrait vraiment appeler une escrime du \Taijijian{} est encore plus rarement évoqué ou pratiqué.

\paragraph*{}
Le présent travail est une tentative de jeter un pont entre la pratique des formes d'épée du \Taiji{} et une escrime du \Taijijian{}.
C'est le résultat de près de quinze ans de recherche et d'expérimentation, à la découverte de la dimension martiale du \Taijijian{}, depuis les notions élémentaires d'escrime, les applications martiales de la forme, jusqu'à la pratique d'assauts libres dans le respect des principes du {T\`{a}ij\'{\i}}. 
Les sources de ce travail sont ancrées dans la tradition du \Yangjia{} \Michuan{} \Taijijian{} transmise par Maître Wang Yen-nien, et en particulier la forme d'épée \Kunlun{}, connue aussi dans ce style sous le nom d'\emph{Épée ancienne}.
L'escrime historique européenne des XIII\textsuperscript{e} au XVIII\textsuperscript{e} siècles a été aussi une incomparable source d'inspiration.
Cela peut sembler curieux de prime abord, mais en réalité, les traités européens d'escrime des siècles passés rappellent parfois nos textes classiques du \Taiji{} et il est frappant de constater combien certaines techniques européennes sont similaires aux mouvements des formes d'épée du \Taiji{}.
En particulier, l'application à la forme d'épée \Kunlun{} du concept de \emph{phrase d'arme} qui décrit les actions d'escrime comme s'il s'agissait d'une conversation, me permit de découvrir des applications martiales convaincantes pour la plupart des mouvements de cet enchaînement.
Lors de ces expérimentations inter-culturelles cependant, j'ai toujours suivi les injonctions des principes du \Taiji{}, des textes classiques et les préceptes de Maître Wang.
Je n'ai pas eu la chance d'être un élève direct de Maître Wang, mais ses livres et mes notes prises au cours des quelques stages où j'ai pu le suivre renferment une mine inestimable d'informations.
Je suis également hautement redevable à mes professeurs qui ont été de ses élèves directs et ont su transmettre fidèlement ses enseignements.
Pour toutes ces raisons, je suis pleinement convaincu que les techniques d'épée et les notions que vous trouverez dans ces pages peuvent être raisonnablement considérées comme des techniques d'épée du \Taiji{} convenables et respectueuses des principes, bien que certaines furent élucidées grâce à des sources indépendantes et néanmoins convergentes.

\paragraph*{}
Il s'agit toujours d'un travail en cours et en évolution constante.
Je ne prétend pas détenir la vérité absolue, ni même avoir réussi à reconstituer de véritables techniques historiques du \Taijijian{}.
Et en réalité, cela m'est égal, ça n'était pas mon objectif. 
La seule chose qui m'importe vraiment est comment ce travail peut améliorer la pratique de l'épée du \Taiji{} et la compréhension des principes du \Taiji{}. 
A ce jour, j'ai le sentiment que ce travail a enfin trouvé sa cohérence et qu'il vaut la peine d'être partagé avec tous ceux qui sont intéressés quelque soit leur style.

\paragraph*{}
Alors que j'envisageai la publication de mes travaux, le faire gratuitement sous la forme d'un site web s'est rapidement imposé comme une évidence.
Cela devrait permettre une large diffusion tout en évitant les tracas de l'impression d'un livre.
Une autre raison importante a été que, avant de publier un livre, il faut bien auparavant le rédiger de la première page à la dernière.
A l'opposé, un site web peut commencer avec un contenu incomplet puis être aisément mis à jour et complété progressivement.
Enfin et surtout, un site web permet une meilleure intégration de media variés tels que les vidéos.
Toutefois, comme une lecture hors-ligne pourrait également s'avérer utile, ce site proposera aussi des versions téléchargeables au format epub et PDF.
Le choix de la license Creative Commons BY-NC-ND répond évidemment aux mêmes
préoccupations de partage.
Les adaptations et la diffusion rétribuée ne sont pas autorisées sans mon accord.
Cependant, le contenu non modifié peut être redistribué librement et gratuitement à condition d'en citer la source.

\paragraph*{}
En espérant que cet ouvrage s'avérera utile aux pratiquants et saura
susciter de nouvelles vocations.

\begin{flushright}
Frédéric Plewniak, janvier 2014.
\end{flushright}
