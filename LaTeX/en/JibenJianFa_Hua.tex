% !TeX spellcheck = en_GB
\section{\Hua}
The verb \Hua{} \begin{CJK*}{UTF8}{bsmi}劃\end{CJK*} means \textit{to delimit}, \textit{to draw}. The same character is also used as a variant of the word for an individual stroke in a Chinese character. 
Along with the fact that the left part of this character is indeed the key for the brush, these observations tend to suggest that the technique somehow evokes the notion of calligraphy, of writing or drawing. 

The emblematic technique is presented in the \Yangjia{} \Michuan{} as a horizontal cut or a large horizontal movement for keeping opponents away. The idea is here to sweep space with the sword to delimit the largest possible area around oneself and slash anyone closing in. 

In a more general perspective, \Hua{} cuts do not need to always be horizontal and encompass a whole range of distances, from very long slashing cuts with the very tip of the blade, to very close-range drawing cuts with the whole edge. In any case, all \Hua{} cuts have in common to be long-energy slicing movements where the blade actually draws a groove in the target instead of splitting it open at once like the short-energy \Pi{} cut does. 

When performing a long-range \Hua{}, the sword is thrown forwards and, when the arm has nearly reached its full extension, slightly before the blade hits the target, the grip is gently tightened to secure the connection between the centres of the sword and the body. The rotation thus continues around the shoulder while the sword is pulling the body forwards until maximal reach is achieved (fig. \ref{fig:hua_cut} a-c). Then, the grip acting as a fulcrum, the sword's inertia pushes back the handle against the heel of the hand. This results in a slicing cut and a backward movement that centres the body back into a guard stance (fig. \ref{fig:hua_cut} d-f).

\begin{figure}[ht]
\centering
	\includegraphics[width=1.00\textwidth]{../../Images/JibenJianfa/Hua/Hua.pdf}
	\caption[Long-range \Hua{} cut]{Long-range \Hua{} cut: (a) Starting from a high position of the sword, (b) the right hand throws the pommel forwards; (c) shows the end of the active phase of the technique; (d) to (f) during the passive phase, the sword's inertia pushes the hand backwards, performing the slicing cut and centring the body back into position.}
	\label{fig:hua_cut}
\end{figure}


At closer range the dynamics of the \Hua{} cut rely less on sword's inertia but more on body structure and movement. Once the edge of the blade is in contact with the target, the slicing cut is generated by pressing the edge against the target while pulling it in a direction parallel to the blade, either with a step or a rotation of the body. On some occasions, in particular when one is passing behind the opponent's back, it may be possible to perform a short range \Hua{} with the false edge. 

Beside being a slicing cut, \Hua{} can also be used to keep the opponents at distance or to incite them to react so we may exploit their action and take control of the rhythm. This is achieved either performing an uncommitted long-range \Hua{} or whirling the sword while advancing. This should definitely be done at a distance close enough to be perceived clearly as a threat even though we may be out of measure when doing so. The ideal distance is actually the very upper limit of the short measure, at which distance, a hit being uncertain yet perfectly plausible, the opponent will feel compelled to react defensively. It is crucial here to get ready to follow up with a more committed attack or with a blade control depending on the opponent's reaction. This second intention will thus ensure to keep the initiative and exploit the opponent's action.

\fiche{
This does not mean though that we must have a predetermined second action in mind, but that we should abandon the initial intention of the \Hua{} and prepare to transform and adapt to the opponent's reaction as soon as they start parrying. This combination of our first uncommitted attack with a timely transformation into a second action while the opponent is just starting to defend against the first one ensures we are one time ahead and keep the initiative. But this definitely requires a state of awareness to avoid an inappropriate action which would allow the opponent to retaliate while we are unprepared. Even though we have taken the initiative, we must none the less conform to the opponent's actions. And we must feel instantly whether the situation is not suitable for a second attack or a riposte, and then withdraw into a safer position out of measure. 
}

While withdrawing after an unsuccessful attack, we may want to keep our opponent away with a series whirling cuts performed in a row, which Master Wang Yennien described as being \Hua{} cuts as well. This application of the technique perfectly fits the translation \textit{to delimit} as it creates indeed a zone of security and prevents the opponent from catching up and attacking us while we are getting out of measure. 

\fiche{
is the principle of the relationship between the centre of the sword and the body.
related to calligraphy: trace a line with the point of the sword thanks to the unity achieved between the sword and the body
NO THE PRINCIPLE OF CONNEXION IS INCLUDED IN SI YAO
the movements of the body are transmitted to the point of the sword which traces lines in the air
}
