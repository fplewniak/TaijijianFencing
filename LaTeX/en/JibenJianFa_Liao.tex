% !TeX spellcheck = en_GB
\section{\Liao{}}
Translating as \textit{to raise}, \textit{to lift}, \textit{to sprinkle}, \Liao{} \begin{CJK*}{UTF8}{bsmi}撩\end{CJK*} is widely mentioned in ancient martial arts manuals for a variety of weapons. The \Yangjia{} \Michuan{} tradition describes this technique as an upward cut, but it is sometimes also considered as a defensive action used to parry or deflect an incoming attack. It is often presented in series of two consecutive cuts or as a combination of a \Liao{} parry immediately followed by an upward cut. 

No precise instructions are given as to whether the cut is a splitting or slashing cut but it seems that both options are equally valid. Technical details will vary according to the cut type or whether \Liao{} is used to parry.
All variations though have in common the upward direction of the movement, as if raising a curtain. This may explain why this \Jianfa{} is designated by a term describing the general movement, \textit{to raise}, rather than the cut itself, contrary to \Ci{}, \Pi{}, \Duo{}, and \Hua{}. In support of this hypothesis,  it is worth noting as well that the \Liao{} character \begin{CJK*}{UTF8}{bsmi}撩\end{CJK*} contains the key for the hand but not the key for the knife.

The basic \Liao{} technique is performed by raising the hand diagonally from a low line towards the opposite upper line in conjunction with waist rotation. (fig)
Depending upon the type of cut involved, slashing \Hua{} or splitting \Pi{}, or whether \Liao{} is combined with \Mo{} or a deflection, the hand movement will extend more or less in the forward direction. 
Whereas deflecting involves hitting the opponent's blade away, a softer engagement allows to gently control and guide the opponent's sword upwards out of the way while keeping our point towards their face to prepare the riposte. In this case, we get contact first with the opponent's blade, then raise the sword while following and controlling the incoming attack. When combining \Liao{} and \Mo{} techniques like this, blade contact is crucial to prepare with minimum risk an appropriate response to the opponent's attack. This contact is not only for protection, it also provides the necessary information to follow the opponent's reactions. 

\Liao{} cuts are oblique, from a low line to the opposite upper line, with the blade aligned with its trajectory and hitting the target in the sagittal line. At the end of the cut, relaxation of the grip allows to recentre the tip and align the blade with the other cutting plane, thus preparing the next \Liao{} cut. This transition is done in front of the body in order to remain protected behind one's blade instead of making a large and open movement. 

A variant \Liao{} cut exists that is done with a sidestep, usually leftwards, to dodge an incoming attack while raising the sword to intercept the opponent's arm. The resulting cut is thus performed laterally rather than in the sagittal plane as described above.

An extreme case lateral \Liao{} cut is performed entirely in the frontal plane, from left to right. It starts first with a rotation of the waist to the left to evade or deflect an attack while preparing the cut. This can be done either in place or with a crossing step to adjust for distance and direction. In both cases we end up facing left, with the opponent at our right. Swinging the sword rightwards while crouching will then aim a \Liao{} cut at the opponent's arm, groin, etc. It is important that the body should be turned slightly towards the right  when crouching so that the right shoulder is not restrained and the sword can be raised high enough to reach and split the target open. 

\fiche{
starting position
inside: little dipper
outside: waist strike
 
but there are many variations for parries depending upon  the action on the opponent's blade. 
}

\fiche{
 insist the blade flat should be strictly following it. 
I.e. cut from left below to right up, or right below to left up, or sideways. 
then the parries and deflections
summarise the general ideas about the hand movements, from the combination with Hua to Pi to Mo, the direction is increasingly backwards, maybe show a figure with a lateral representation which will be less awkward than obscure textual explanations
}


\fiche{
Starting on the left leg with the right foot forward and the sword in the left lower line, the weight is shifted onto the right foot with a rotation of the hips, which gives the initial impetus to the sword. Then, power is released, raising and accelerating the sword with an expansion of the body and a forward extension of the arm. The blade follows a slanted trajectory from the lower left to the upper right line, sliding along the target, or traversing it in the sagittal plane in the case of a splitting cut. When performing a splitting \Liao{}, just as was described for \Pi{}, once the sword's centre of gravity has passed the hand, no more action should be exerted on the handle to let the sword fly freely through the target. The body should be relaxed to follow the sword, get the energy of the sword back after the cut and recentre oneself in a ready position.

From this position it is possible to let the sword continue its movement backwards down to the lower right line while passing the left foot forward, to prepare a second \Liao{} from right to left. In order to do the second cut, the weight is first transferred onto the left foot, then the rotation of the waist draws simultaneously the sword and the right leg forward. The blade thus follows an oblique trajectory from lower right to upper left either slashing or splitting the target.



\Liao{} can also be used for parrying and, as a matter of fact, the \Liao{} cuts described above can all be preceded by a 
\fiche{ It is worth noting that for some styles, \Liao{} refers exclusively to this type of raising the opponent's sword aside and does not involve cutting in itself. (check that with Mattias)}
}

\fiche{
	It is more difficult as well to gather the energy back directly, but actually we can play with gravity which will help slowing or stopping the sword in a high position from which we may start a new technique by letting the sword go with its own weight and momentum,  using its centre of gravity as a fulcrum. 
	check where is the cut located depending on the inclination of the movement
	
	The energy generating the movement comes from the expansion of the body and waist rotation.
	This is more apparent when performing \Liao{} in a standing position, which needs less to rely on the recycling the sword's inertia after the previous move.
	
	the body is drawn forwards, the foot follows the sword with some delay. The whole movement is actually a centring passive action. The sword does the job
	
	combination of a splitting upward cut followed by a drawing cut. During the latter,  
	when cutting behind, there is only a splitting cut
	the raising m  ovement may be combined with either a splitting cut or a drawing cut
	crouching only splitting type (not quite sure as it may depend upon the distance)

}

