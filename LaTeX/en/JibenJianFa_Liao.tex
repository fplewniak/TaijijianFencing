% !TeX spellcheck = en_GB
\section{\Liao{}}
Translating as \textit{to raise}, \textit{to lift}, \textit{to sprinkle}, \Liao{} \begin{CJK*}{UTF8}{bsmi}撩\end{CJK*} is found in various ancient manuals for different weapons. The \Yangjia{} \Michuan{} tradition describes the technique as an upward cut, but it is sometimes also considered as a defensive action used to parry or deflect an incoming attack. Technical details will vary according to the type of cut performed, splitting or drawing upward cut, or whether \Liao{} is used to parry.
All variations though have in common the upward direction of the movement, as if raising a curtain, which may explain the fact that this character has the key for the hand instead of the knife.

In the \Yangjia{} \Michuan{} tradition, \Liao{} cuts is usually presented in series of two consecutive cuts, one from left to right then one from right to left.  

\fiche{
Starting on the left leg with the right foot forward and the sword in the left lower line, the weight is shifted onto the right foot with a rotation of the hips, which gives the initial impetus to the sword. Then, power is released, raising and accelerating the sword with an expansion of the body and a forward extension of the arm. The blade follows a slanted trajectory from the lower left to the upper right line, sliding along the target, or traversing it in the sagittal plane in the case of a splitting cut. When performing a splitting \Liao{}, just as was described for \Pi{}, once the sword's centre of gravity has passed the hand, no more action should be exerted on the handle to let the sword fly freely through the target. The body should be relaxed to follow the sword, get the energy of the sword back after the cut and recentre oneself in a ready position.

From this position it is possible to let the sword continue its movement backwards down to the lower right line while passing the left foot forward, to prepare a second \Liao{} from right to left. In order to do the second cut, the weight is first transferred onto the left foot, then the rotation of the waist draws simultaneously the sword and the right leg forward. The blade thus follows an oblique trajectory from lower right to upper left either slashing or splitting the target.

Variants of the above cuts exist that are done with a sidestep to dodge an incoming attack while raising the sword to intercept the opponent's arm. The resulting cut is thus performed laterally rather than in the sagittal plane as described above. 

An extreme case of such a lateral \Liao{} cut is performed entirely in the frontal plane, from left to right. It starts first with a rotation of the waist to the left to evade or parry an attack while preparing the cut. This can be done either in place or with a crossing step to adjust for distance and direction. In both cases we end up facing left, with the opponent at our right. Swinging the sword rightwards while crouching will then aim a \Liao{} cut at the opponent's arm, groin, etc. It is important that the body should be turned slightly towards the right  when crouching so that the right shoulder is not restrained and the sword can be raised high enough to reach and split the target efficiently. 

\Liao{} can also be used for parrying and, as a matter of fact, the \Liao{} cuts described above can all be preceded by a \Liao{} parry to deflect or raise the opponent's blade out of the way. Whereas deflecting involves hitting the opponent's blade away, a softer engagement allows to gently control and guide the opponent's sword upwards while keeping our point towards their face to prepare the riposte. In this case, we get contact first with a \textit{receive} technique, then raise the sword and follow the incoming attack. The contact of the blades is crucial here to guide the correct positioning of the body and prepare the riposte conforming with the attack with minimal risk.
\fiche{ It is worth noting that for some styles, \Liao{} refers exclusively to this type of raising the opponent's sword aside and does not involve cutting in itself. (check that with Mattias)}
}

\fiche{
	It is more difficult as well to gather the energy back directly, but actually we can play with gravity which will help slowing or stopping the sword in a high position from which we may start a new technique by letting the sword go with its own weight and momentum,  using its centre of gravity as a fulcrum. 
	check where is the cut located depending on the inclination of the movement
	
	The energy generating the movement comes from the expansion of the body and waist rotation.
	This is more apparent when performing \Liao{} in a standing position, which needs less to rely on the recycling the sword's inertia after the previous move.
	
	the body is drawn forwards, the foot follows the sword with some delay. The whole movement is actually a centring passive action. The sword does the job
	
	combination of a splitting upward cut followed by a drawing cut. During the latter,  
	when cutting behind, there is only a splitting cut
	the raising m  ovement may be combined with either a splitting cut or a drawing cut
	crouching only splitting type (not quite sure as it may depend upon the distance)
}
