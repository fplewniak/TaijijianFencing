% !TeX spellcheck = en_GB
\chapter{\Taijijian{} practice}
\Taijijian{} is the pinyin transcription for the Chinese word designating the art of the sword based on the \Taiji{} principles \textendash{} also known as the \Taiji{} (or Taichi) sword \textendash{}  and studied together with \Taijiquan{}, the \Taiji{} boxing.

In most styles, \Taijijian{} practice consists essentially, if not exclusively, in learning and performing a sword form.
As such, it complements bare hand practice and provides the practitioners with an invaluable tool to improve their skills.
The sword is indeed a devoted partner, always ready to guide us on the way towards a better understanding and embodiment of the \Taiji{} principles.

However, I am convinced that, in this respect, \Taiji{} sword practice can only benefit from the thorough study of basic techniques, fencing notions, partner drills, martial applications and even free play.

In the XXI\textsuperscript{st} century, \Taiji{} fencing does not have to be practical any more: fighting skills and efficiency are not sought for themselves, but their purpose is nothing but the application of the \Taiji{} principles in action. From basic techniques to free fencing, the practitioners will strive to achieve unity with their sword, improve their nimbleness and sensitivity, relax their body and mind, develop their \Yi{}, etc. 

Last but not least, this life-long endeavour should also bring its share of fun.

\section{\index{basic exercises}Basic exercises and \index{warm-up}warm-up}
The goal of warm-up and basic exercises is the preparation of the body and the mind to the safe performance of efficient techniques conforming to the \Taiji{} principles.

Warm-up should gently mobilise the joints and muscles to bring them to their optimal functioning level for smoother movements and lower risks of strains.
An emphasis on the upper body is necessary, in particular the shoulders, arms, wrists and fingers, which do most of the job of sustaining the sword.
But the lower body should not be overlooked in preparation to the footwork which is of crucial importance in fencing.

I usually tend to start the warm-up by mobilising the pelvis and the spine, which are at the core of every movement in \Taiji{}, and then continue with the shoulders, elbows, fingers and wrists.
For the lower body, I start with the ankles, carry on with the knees and then the hips, exercising balance at the same time.
Eventually, the session ends with codified and free footwork.

Before proceeding to basic techniques, it might be a good idea to finish the warm-up session using the sword as an accessory.
This will not only serve to further stretch the wrists and shoulders, it will also help the practitioners to exercise their relationship with the sword.
The repetition of a simple non-technical movement helps indeed to install the relationship between the sword and the body: the sword's energy, absorbed and concentrated in the body, is returned to the sword for the next movement\footnote{See chapter \ref*{ch:grip} for more details on handling the sword.}.

Repetition is of crucial importance as well for practising basic techniques, the emblematic basic cuts and thrusts. Repeating them in series is essential to technical precision and body control, requisite conditions for the proper realisation of techniques in less controlled situations.

It is interesting as well to practise the basic techniques with \index{target}targets, not only to exercise precision, but also to develop the mindset and relaxation of the body appropriate for an effortless efficiency of techniques.
For \index{thrusts}thrusts, \Ci{} or \Zha{}, cloth-pegs hung on a string at throat level make pretty convenient targets.
Targets for \index{cuts}cutting techniques are more difficult to set up: cuts are supposed to pass through the target, hence hard resisting targets are out of question. Non ligneous plant stalks or a slab of clay can make appropriate cheap targets for test cutting without requiring a dangerously sharp blade. Clay works quite nicely with a regular blunt sword, but do not use your favourite one or be prepared for intensive cleaning of your sword after the session. Make sure as well that clay does not get soiled with pebbles or sand to avoid damaging the blade.

Those exercises build up the basics which in combination constitute the source of all the techniques that can be practised in the more diverse context of the form.


\section{\index{form!practice@\textit{practice}}Form practice}
The form is a set of movements arranged in a continuous succession of techniques which some practitioners present as a mock fight with an imaginary opponent.

However, I personally think that this is not the complete story: although the movements of the form are martial techniques indeed, the whole set does not constitute a single combat from start to end. Techniques are rather arranged in short series, which the European tradition calls 'pieces', describing a variety of situations where these techniques may be applied. Variations are given throughout the form and as most movements may have several applications, series may overlap or describe varying situations.

The form is much more than a catalogue of techniques, it is the source from which all applications proceed according to the infinity of possible situations. As an actualisation of the \index{YinYang@\Yin{}/\Yang{}}\Yin{}/\Yang{} principle issuing from the \Taiji{}, the \Taijijian{} form contains all the potentialities for the generation of countless applications. Performing the form is thus generating a potential for infinite creativity, reserving the practical expression of techniques for their application in real situations.

In my opinion, the form is therefore essentially a tool for achieving a better understanding and embodiment of the \Taiji{} principles. 
Memorisation is only a beginning: what is truly trained when practising the form is \index{YinYang@\Yin{}/\Yang{}}\Yin{}/\Yang{} \index{transformation}transformation and directing the \index{Yi@\Yi{}}\Yi{} as a continuous yet ever-changing flow.
Gradually, with practice, unity with the sword can be achieved, the form becomes more and more internal, gains in fluidity, feels easy and natural. 
Until ultimately, I like to think that the body and the mind are delivered from all their tensions, and nothing remains but pure \index{Yi@\Yi{}}\Yi{} effortlessly generating the form.

The form, however, is not only a mental exercise but it also physically trains the body.
Some movements have indeed an exaggerated amplitude to develop strength, balance and flexibility.
Others are clearly intended to be spectacular and demonstrative. 
Every \Taijijian{} form is thus made of a mixture of internal martial training, gymnastic exercise, and spectacle.
Discerning how these characteristics are actually expressed in the different movements allows the practitioner to favour at will one aspect or the other.

To those interested in improving their fencing skills, the form provides an invaluable tool for building a strong repertoire and knowledge of martial techniques while practising the body mechanics appropriate for their effective application.

The form, however, is not strictly representative of the occurrence of techniques in free fencing, where the most frequent and effective ones are rather simple.
The form contains surprisingly complex movements that may contribute to its demonstrative character, but probably also prepare the practitioner to master extreme techniques appropriate to exceptional circumstances.  
Most swordsmen of the time might well have never needed to put these techniques to practice, those who actually had to might have owed their life to this preparation instead of having been overwhelmed by a stunning situation.

In modern times, martial efficacy of \Taiji{} fencing is not a matter of survival any more, but, within the context of friendly controlled practice, aims more readily at freeing the mind and the body from their tensions.
The martial interpretation of the form thus provides a whole range of situations where the practitioners may put their body and mind to the test in partner drills and martial applications.

\section{\index{two-person drills!practice@\textit{practice}}Two-person drills and \index{martial applications!practice@\textit{practice}}martial applications}
Two-person drills are simplified and codified situations whose goal is to introduce and exercise important principles and notions of \Taiji{} fencing such as distance and time, the lines, sensing through the blade, footwork in response to the opponent's moves, nimbleness, etc.
Entirely dedicated to this pedagogic goal, those drills are essentially continuous exercises or games, most of the time without much concern for the realism of the situations.

Martial applications will then develop the same principles and notions within a codified or semi-codified simulated piece of fight, simultaneously giving examples of the potential practical use of techniques from the form.
As such, they open the way to a better understanding of the form, highlighting the   martial essence of movements and the distinction between practical and demonstrative moves.

Although martial applications may be much more realistic than two-person drills, it must be acknowledged that they none the less are simulations far from reproducing exactly all aspects of a true sword fight to death.
Depending on their experience and protection level, practitioners may perform the applications at different speeds or may be more or less well-disposed towards their partner, thus achieving different levels of realism. 
Some applications may work well at low speed, with caring partners but not any more when performed faster, with partners who do not hold their attack. 
Performing applications with full \index{protective gear}protective gear which allows full blows may thus be more demanding and what used to be working with less protections and more precautions might not work as nicely.
On the contrary, performing an application at low speed may allow a non-cooperative partner to counter-attack whereas such a riposte would have required light-fast reactions against a technique performed at full-speed. 

Evaluating the true effectiveness of martial applications needs therefore to account for all those factors.
For what is worth, as far as they allow us to develop and practise the \Taiji{} principles and fencing notions, the approximate realism of martial applications should give entire satisfaction for our purpose. The efficiency of an application should be only a consequence of our conformance to the principles, and definitely not a matter of speed or strength.
Thus, martial application practice may help develop the proper mindset and body disposition for the effective application of techniques and principles in free play.

\section{\index{free play!practice@\textit{practice}}Free play}
Free play refers to the wholly non-codified simulation of a sword fight.
Depending on the protective gear worn by the practitioners and their experience, rules may be defined to guarantee their safety.
In any case, violence is definitely ruled out and free play should always remain a friendly game, without any overly competitive mind.

I understand that some practitioners may feel concerned that free play might possibly not be considered as internal.
Actually, expressing martial techniques should not be confused with external practice. 
My personal view is that we may speak of \index{internal practice}internal practice as long as every movement is born from the \index{Yi@\Yi{}}\Yi{} which shapes the technique and, originating from the centre, is eventually expressed towards the periphery. Efficient techniques proceed naturally from the appropriate intention and a relaxed body, well mastered principles that have become natural and can thus be applied spontaneously.
Students of internal arts should not cling too much to trifling technical details, which are only the finger of the wise man. 
They should reach for the moon: develop their capacity to apply the principles in challenging and unpredictable situations.\footnote{To be honest, I do not even think there are that many differences between internal and external arts when it comes to high level practice. The main difference would rather lie in the way these arts are taught. External arts first focus on techniques and let the student figure out the principles whereas internal students are taught the principles and must figure out how to perform the techniques in conformance.} Everything else should follow.

Nimbleness of the \index{Yi@\Yi{}}\Yi{} and of the body results from an open and free, tensionless mind allowing us to remain relaxed when facing the threats of an opponent.
After all, this may be the true practical application of martial arts in our modern times: not to let ourselves overwhelmed by stress in all matters of urgency.

Of course, as already mentioned above, technique efficacy is entirely relative to the context: free play is \textendash{} and must stay \textendash{} no more than a simulation that cannot reproduce all aspects of a true sword fight, and in particular the psychological aspects.

In any case, nowadays, arguments are not settled any more in duels or sword fights and the purpose of \Taiji{} fencing is more a matter of personal development than of actual fighting capacities. Our main concern is not to hit the opponent by all means but to do it with the appropriate manner while not being hit. How the goal is reached is more important than the goal itself: scoring a hit against an opponent should be the result of the proper application of the \Taiji{} principles to the current situation, and not a purpose in itself.

Thus, in order to avoid excessive competition,  I prefer not to count hits and only appreciate subjectively the quality of the actions. Taking videos of free play sessions may also help reviewing actions afterwards to highlight the positive and the negative. 
With really nothing at stake, this less competitive approach allows to focus more on the principles and internal practice and limits the risks of accidents.

\fiche{
Not quite sure about the following, should be confirmed before including it:
This goal revives somehow the Neo-Confucean ideal of self-cultivation which arose during the \Ming{} dynasty and was expressed by the literati, among other things, in their perception of martial arts.
}

\fiche{
\section{Internal vs. external}
    movement coming from the centre, associated with yi which will shape the technique, expression towards the outside. should not attach too much importance to technical details which are only a consequence of the correct performance and application of principles, being relaxed and guiding from the centre with the yi but adapted to external energy, link and sensitivity.
    external does the opposite, the technique us guided by the periphery of the body.
        
}

\section{\index{safety}Safety considerations}
The practice of \Taiji{} has been associated with health and personal development for perhaps over a century now. We could thus expect that preserving a good health should go along with the preservation of our physical integrity, and that \Taijijian{} practice could be taken rather safely, from solo form to free play.
Actually, even hard training of warriors in the past may have presented some degree of safety: what would have been the point of decimating the troops before actually sending them to battle. 

Of course, accidents sometimes happen, but there is no reason why they should be the norm. We should in all circumstances bear in mind that even a blunt sword can be a deadly weapon and we should behave accordingly. 
Safe practice actually results from the combination of a responsible attitude with the appropriate equipment. 
The mind-set we adopt during any kind of practice is most important indeed. It is essential to me that we feel responsible not only for the physical integrity of other people around us when we hold or wield a sword, but also for our own safety. 
The first consequence of this state of mind is that, if it is adopted by all practitioners, everyone constantly maintains a good degree of watchfulness instead of solely relying on others for their safety.
Furthermore, if we ever get hurt despite all precautions, this attitude also prevents us from systematically blame others for it.

It is also important to remember that solo practice is not exempt from danger. Our concern for safety must not be limited to the practice ground but starts in the changing-room. As soon as we start holding or wielding a sword,  we must do so most carefully. When carrying a sword, its blade should be held vertically along the arm or its tip should be pointing to the ground. Never ever wave a sword heedlessly and always make sure to be at safe distance from other people before starting to practise. 

When it comes to \index{two-person drills!safety@\textit{safety}}partner drills, \index{martial applications!safety@\textit{safety}}applications or \index{free play!safety@\textit{safety}}free play, we should always adapt the speed and intensity of the practice to the less advanced partner and to the protective gear. 

Both partners should use the same kind of equipment: a blunt sword\footnote{See chapter \ref*{ch:chinesesword} for more details.} and a \index{protective gear}fencing mask are a minimum. Gloves and a padded jacket allow more dynamic practice and are highly recommended for free play. 

If the above advices are followed, the risks of accident may be kept at a minimum. However, it should be always remembered that there is an inherent risk to any martial practice, that accidents can happen, and that when they do, the only thing we can do is to minimise as much as possible their consequences with the appropriate equipment and attitude. Those who partake in \Taijijian{} fencing should acknowledge this idea and accept that they do so at their own risk. 

\fiche{ keep some sense of responsibility for any incident that may happen
 and urges us to seek the mistake in our acts first. We may thus always learn lessons from accidents and keep a friendly spirit with other practitioners. 
}

\fiche{
Maybe use the following in the chapter about handling the sword.

no constraint on the sword, leverage the weight and inertia of the sword to move it.

Li strength vs Jin force, Jin is a the use of the whole body to express strength in a connected and tensionless manner. 
Require awareness of the situation: position, direction and movements of the sword, to be able to adjust and move it without tension.

\Taiji{} insists on not using muscular strength (actually not more than required to stand and hold the sword) Chinese uses the \Li{} word (physical strength, strive) as opposed to {J\`{\i}n} (physical strength as well, but vigour, expression, energy), 
\Li{} seems to be less subtle than {J\`{\i}n}
It is important to note at this point that the connection between the body and the sword is indeed a two-way relationship. While it is true that one's sword should become an extension of one's arm, this claim is basically of no help at all to most {T\`{a}ij\'{\i}j\`{\i}an} beginners who desperately try to figure out how to cope with a 30-inch-long blade. I have found much more informative to liken the sword to a dancing partner. 
    In the same way, a good sword player should make her or his sword look as if it were alive, with graceful and fluid movements. There should be no hard constraint placed upon the sword, nor should the body be strained in any way by the sword inertia. 
This can only be achieved by a constant awareness of the sword position and movements and by a good relaxation allowing to absorb the energy back from the sword and perform the next move accordingly. become one with the sword (corollary of sword being an extension of body and principles: body is united)
 
}
