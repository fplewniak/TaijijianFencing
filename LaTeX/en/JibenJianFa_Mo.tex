% !TeX spellcheck = en_GB
\section{\Mo}
\Mo{} \begin{CJK*}{UTF8}{bsmi}抹\end{CJK*} encompasses all the techniques that take control of the centre of the opponent's blade.
\fiche{
	principle of controlling, deflecting, directing, listening
	the intention is located closer to the forte
	
	translation: put, spread (butter, etc.), wipe
	use the simple character
	
	maybe related to a variety of techniques used to parry, such as wash, shave, shear. ..
	
	radical mo is probably phonetical although it means powder (waste, residuals)
	
	sliding contact of the blades (wipe) with a deflecting action of control thanks to the opposition of the forte against the feeble of the opponent,  this is found in blade capture (coulé, opposition, ...) or the control of the blade during attacks
	
	lateral and aiming at the centre of the opponent's blade, if only lateral we give way to transformation by the opponent and we don't have control of their blade. 
	
	relationship with feeling through the blade and cover
	during parries, absorption, deflection, protection
}
