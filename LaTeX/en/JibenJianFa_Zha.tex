% !TeX spellcheck = en_GB
\section{\Zha}
In the \Yangjia{} \Michuan{} sword tradition, \Zha{} \begin{CJK*}{UTF8}{bsmi}扎\end{CJK*} is presented as a downward thrust. As a matter of fact, \Zha{} means to set up, a tent for instance, to poke, to plug-in, etc. All kind of actions which generally follow a downward direction. However, in the \Yangjia{} \Michuan{} perch form list of movements, both verbs \Zha{} and \Ci{} are used jointly to describe horizontal thrusts. We may thus argue that \Ci{} designates thrusts in general,  as said above,  whereas \Zha{} is a particular case thereof,   usually pointing downwards, but sometimes horizontal. 

The main characteristic of \Zha{} is that the power of the thrust comes from the inward rotation of the waist with the weight on the forefoot instead of from an expansion from the hind leg.   The form of this movement is very similar to the \Tuishou{} movement of the first section of the Long form. (fig)
If we cling strictly to this definition, then, some short \Ci{} described above may be considered as a \Zha{}. These movements are indeed quite similar to the thrusts performed with the perch. (fig)

While performing a \Zha{}, the back is stretched, the breast is relaxed and the arm is extended in coordination with the waist rotation. This creates a spiralling arc of force from the foot to the sword point, passing through the back, along the outside of the arm and along the true edge. Art close range,  the unarmed hand may simultaneously control the opponent's arm, to avoid a counter strike and to open the way for the thrust.

\fiche{
what about punches? 
	
	rem: there exists another character which is simplified into this one and is pronounced at the first tone and means
}