% !TeX spellcheck = en_GB
\chapter{Preface}

\paragraph*{}
In most styles and schools, \Taijijian{} is studied quite exclusively as routines and most of the time without much consideration for its martial roots. 
Although we see now a growing interest in two-person drills, mainly in the form of sticking sword exercises, martial applications are very rarely demonstrated and are often limited to martial explanations and justifications of the routine movements. 
What may be truly called \Taijijian{} fencing is even more rarely evoked or practised.

\paragraph*{}
The present work is an attempt to bridge the gap between \Taiji{} sword form practice and \Taijijian{} fencing.
It is the result of nearly fifteen years of research and experimentation,
trying to uncover the martial dimensions of \Taijijian{}, from fencing basics and martial applications of the form through free sparring with respect for the \Taiji{} principles.
The sources of this work are rooted in the \Yangjia{} \Michuan{} \Taijijian{} tradition as transmitted by Master
Wang Yen-nien, and mainly the \Kunlun{} sword routine, also known in
this style as the \emph{Old Sword Form}.
Historical European fencing from the XIII\textsuperscript{th} to the XVIII\textsuperscript{th} centuries also provided valuable inspiration.
This might seem strange at first sight, but actually, European fencing treatises of past centuries do remind sometimes of our \Taiji{} classics and a good deal of European techniques present striking similarities to those found in \Taiji{} sword forms.
In particular, applying to the \Kunlun{} routine form the
concept of \emph{fencing phrase} which describes fencing actions as if
it were a conversation, allowed me to discover convincing martial
applications for most movements of this form.
In all those cross-cultural experimentations though, my guides have always been the \Taiji{} principles, \Taiji{} classics and Master Wang Yen-nien's precepts.
I did not have the luck to be a direct student of Master Wang, however his
books and my notes from the few workshops I could attend contain a
wealth of enlightening information.
I also owe my teachers, who had been his direct students and truthfully transmitted his teachings, an immense debt of gratitude.
I do believe therefore that the sword techniques and notions you will find in these pages can confidently be considered as appropriate \Taiji{} sword techniques, respectful of principles, even though some may have been elucidated thanks to unrelated but nonetheless convergent sources.

\paragraph*{}
This is still a work in progress and in constant evolution.
I do not pretend to hold the absolute truth nor to have succeeded in
reconstructing genuine historically accurate \Taiji{} sword techniques.
Actually, I do not care, this was not my goal.
The only thing that really matters to me is how this work can help improving \Taijijian{} practice and the comprehension of \Taiji{} principles.
Today, I have the feeling that this work is consistent at last and worth sharing with all those interested no matter the style they practice.

\paragraph*{}
As I considered publishing my work, doing so for free in the form of a
web site soon became self-evident.
It would ensure a wide diffusion while at the same time avoid all the hassle of book printing.
Another important reason was that, before you can actually appear in print, you have to write the book from beginning to end first.
A web site, on the opposite, can start with partial content and may be easily updated and augmented gradually with new material.
Last but not least, a web site would allow better integration of various media such as video.
However, since off-line reading might be desirable sometimes, downloadable
versions will also be made available in epub and PDF formats.
The choice of the Creative Commons BY-NC-ND licence was also an evident response to the same concerns.
Derivative works and commercial distribution are not allowed without my prior consent but unmodified content may otherwise be freely redistributed provided proper credit is given.

\paragraph*{}
Hoping that this work will prove itself useful to practitioners and may
spark off new vocations.

\begin{flushright}
Fr\'{e}d\'{e}ric Plewniak, January 2014.
\end{flushright}
