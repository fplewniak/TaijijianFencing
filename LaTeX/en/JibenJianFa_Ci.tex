% !TeX spellcheck = en_GB
\section{\Ci{}}
The word \Ci{} \begin{CJK*}{UTF8}{bsmi}刺\end{CJK*}, which means \textit{to thrust}, \textit{to pierce}, \textit{to stab}, is used in the \WubeiZhi{} as a generic term referring to all thrusting techniques. It is mentioned as well in other ancient treatises to describe thrusts with a variety of weapons.

In the \Yangjia{} \Michuan{} tradition, \Ci{} can be defined as a powerful upward or horizontal thrust where the point is pushed forcefully through the target. 

The formal technique is habitually performed starting on the right foot, left leg forward, either with a passing step (long \Ci{}) or with a simple transfer of the weight onto the left foot (short \Ci{}). In the formal context of drills and routine practice, the short \Ci{} is aimed at the belly and the long one at the throat. When sparring though, other parts of the body, such as the torso or even the face, are also targeted.

Long or short, the technique invariably starts by creating in the body a spiral structure connecting the left foot to the sword. As soon as the waist starts moving, the right arm pushes on the handle and rises in a spiralling movement that ends up with a flat horizontal sword position, the pommel oriented towards the left hip. Simultaneously, the weight is transferred onto the left foot. The grip gradually adapts to achieve a uninterrupted connection between the hand and the handle, without any kink, suitable for pushing the sword forwards effortlessly. The adjustment of the grip also permits the exertion on the handle of an oblique action reaching through the guard for a point just beyond to generate at the tip a pivot point that stabilizes it\footnote{See chapter \ref*{ch:chinesesword} for more details on pivot points.}. 

In the long version of the technique, a greater reach is achieved thanks to a passing step of the right foot. The right arm must be extended before stepping forward in order to improve the precision of the thrust and to keep the body as far as possible from danger behind the sword. Further protection can also be achieved by binding the opponent's blade to control it with the guard or the forte.

\begin{figure}[ht]
\centering

	\includegraphics[width=1.00\textwidth]{../../Images/JibenJianfa/Ci/Ci_thrust_arc.pdf}
	\caption[Long \Ci{} thrust]{At the end of the long \Ci{} thrust, the sword is aligned with the left hip but its point is centred, aiming at the base of the throat. The power of the whole body structure is concentrated into the sword to forcefully push the tip through the target.}
	\label{fig:ci_thrust}
\end{figure} 

Ideally, the right heel should touch the ground exactly at the same time as the blade tip reaches the target. The relaxation of the structure then completes the passing step while pushing the blade through. While doing so, it is important not to fall into the right leg to keep our ability to withdraw quickly if needed. This does not mean though that the weight should not be in any way transferred onto the right leg, but that the polarity empty/full between the two legs should be maintained under all circumstances so as to avoid double weight. A powerful yet mobile structure is thus achieved by the generation of an arc of force, going from the left foot, traversing the back, spiralling along the right arm to reach the tip of the blade, and backed up by the spiral in the left arm and sword fingers.

Besides the above emblematic form, the \Ci{} techniques may encompass other powerful thrusts leveraging the body structure to push the sword forwards in a, clockwise or anticlockwise, spiral. In all those techniques, a protective cone is created,  whose point aims at the target and within which one can step in, safely hidden behind one's own sword. 

\fiche {
figures: 1 hand position, spiral path in the whole body,  protection arm extended

3 the protective angle of the guard is increased when it is farther from the body

This way of controlling the opponent's blade bears some similarity to the capturing of the opponent's blade with the guard mentioned in the Principles of the Wudang sword.
It should be noted here too that this kind of control is entirely consistent with the fact that the tip is a pivot point when applying a force on the blade near the guard of a properly balanced sword. This ensures indeed that the contact of the opponent's blade has virtually no effect on the tip and does not prevent it from staying in line with the target when thrusting. 
}